\chapter{Introduction}
\label{c:introduction}
Particle physics research over the last century or so has provided us with our current basic understanding of
the fundamental particles that have made up the universe since its origin, and their interactions with each
other. This is summarised by the Standard Model (SM) of particle physics, which has been developed
incrementally in recent decades and has stood up well to scientific scrutiny. The Large Hadron Collider (LHC)
at the Organisation Europ\'{e}enne pour la Recherche Nucl\'{e}aire (CERN) near the Swiss city of Geneva
(Figure~\ref{fig:LHC_map}) was constructed with the aim of investigating the SM. Areas of current interest
including electroweak symmetry breaking, the Higgs mechanism and physics beyond the SM (BSM) such as
supersymmetry (explained in further detail in Section~\ref{c:the_standard_model}), require the acceleration of
particles to high energies (of the order of several TeV). The start of data-taking from proton-proton
collisions at the LHC in 2009 ushered in a new era in terms of energies at particle colliders, taking over as
the highest energy particle collider from the TeVatron at Fermilab, near Chicago, U.S.A. In 2010 and 2011 the
LHC collected data at a centre-of-mass energy of 7\TeV (5.1\fbinv), followed by 8\TeV (21.8\fbinv) in 2012 and
currently in 2015, after the first long shutdown, at 13\TeV.

The Compact Muon Solenoid (CMS) general-purpose detector is one of the four main detectors located around the
LHC (the others being ATLAS, LHCb and ALICE), approximately 100\m below ground level.

\begin{figure}[!hbtp]
   \centering
     \includegraphics[width=0.6\textwidth]{Chapters/01_Introduction/Images/lhc-pho-1997-169.jpg}\\
     \caption{Map of LHC location.}
     \label{fig:LHC_map}
\end{figure}

This thesis presents an analysis based on the full proton-proton collision data from the CMS experiment in
2011 and 2012. The analysis investigates the top-antitop (\ttbar) differential cross section with respect to
global level event variables, specifically in semileptonic \ttbar decays in the electron+jets and muon+jets
channels. This investigation is motivated primarily by the importance of understanding \ttbar events, since
they are a significant background in many new physics analyses. The understanding of event generators and QCD
events provided by studies such as this is also helpful for other physics analyses. Furthermore, rare Standard
Model processes and new physics scenarios could be detected in the distributions of the global variables
under investigation.

The work presented here was carried out in collaboration with Emyr Clement, L{}ukasz Kreczko, Sergey Senkin
and Philip Symonds under the supervision of Professors Joel Goldstein and Greg Heath. The main contribution of
the author to these studies lay in development of the physics analysis, implemented in a C++ and Python
software framework, playing the leading role in producing the combined 7\TeV and 8\TeV analysis described in
Chapters~\ref{c:Differential_Cross_Section:data_simulation_and_selection},
\ref{c:Differential_Cross_Section:fitting_unfolding_and_measurement}
and~\ref{c:Differential_Cross_Section:systematics_and_results}. Areas of focus regarding physics included
event selection synchronisation of the event selection, distribution comparison between 7\TeV and 8\TeV data,
and implementing details such as selection criteria, \btagging, trigger efficiencies, jet energy resolution
and fitting, in addition to maintaining up-to-date particle object definitions, corrections, efficiencies and
prescription recommendations from working groups within CMS.
% In terms of the technical workflow employed, the author was heavily involved in producing ntuples from the
% analysis-ready Analysis Object Data (AOD) data format, running the software to perform the prescribed
% analysis methods, and running scripts on the output ROOT data to perform the final calculation and to
% produce results plots and tables.

Chapters~\ref{c:CMS_Detector} and \ref{c:CMS_computing_and_offline} describe the LHC and the CMS detector,
including information about the object reconstruction process based on detector readout, to represent
particles produced in collisions. Chapter~\ref{c:the_standard_model} provides an overview of the Standard
Model theory and some of its shortcomings, followed by a review of physics of the top quark at the LHC in
Chapter~\ref{c:top_physics_at_the_lhc}. A small study investigating the \btagging algorithms used in CMS is
described in Chapter~\ref{c:b_tagging_study}, and the main \ttbar differential cross section analysis is then
covered in Chapters~\ref{c:Differential_Cross_Section:data_simulation_and_selection},
\ref{c:Differential_Cross_Section:fitting_unfolding_and_measurement}
and~\ref{c:Differential_Cross_Section:systematics_and_results}. To conclude,
Chapter~\ref{c:summary_and_outlook} contains a summary and outlook to the future. Additional data, tables and
plots from the presented analyses are given in Appendices~\ref{ac:b_tagging_plots} and \ref{ac:ttbar_diff_cross_section_analysis}.

From the outset, natural units are used throughout this thesis, unless otherwise specifed, so that
\begin{equation}
\hbar = c = 1,
\end{equation}
meaning that mass, momentum and energy all have the same units of electronVolts (eV).