\chapter{Summary and Outlook}
\label{c:summary_and_outlook}

This thesis has presented an overview of the theoretical background to the Standard Model, a summary of the
CMS detector at the LHC, a small scale study of \btagging algorithms employed in CMS physics analyses, and a
measurement of normalised differential \ttbar cross sections with respect to global variables \met, \HT, \st,
\mt and \wpt in proton-proton collisions with 5.0\fbinv of data at \roots=7\TeV and with 19.7\fbinv of data at
\roots=~8\TeV, collected with the CMS detector.

The results of this analysis confirmed the dpreviously observed characteristic \cite{Chatrchyan:2012saa} of a
\pt distribution that is softer in data than in the simulation from the \MADGRAPH, \MCATNLO and \POWHEG
generators. The simulated \MADGRAPH distribution, which is used as the nominal signal sample in this analysis,
once corrected for this mismodelling shows good agreement with data. Otherwise, the data shows good general
agreement with predictions, showing that these commonly used Monte Carlo simulation generators can be used
with confidence to model \ttbar events. The \MCATNLO generator performs well at $\roots=8\TeV$ in terms of
agreement with measurements, with \POWHEGPYTHIA2 also predicting data well at both $\roots=7\TeV$ and
$\roots=8\TeV$. In Run 2 CMS top analyses at $\roots=13\TeV$, samples generated with \POWHEGPYTHIA v8 are used
as the nominal \ttbar signal sample, in which the top \pt mismodelling has been corrected. %TODO check this.

Whilst the tendency for maximum deviations of observations from predictions is in the tails of the
distributions, the general trend is for simluations to provide overestimations of the data. This appears to
rule out signs of rare SM processes or BSM physics described in
Sections~\ref{s:Incompleteness_of_and_physics_beyond_the_SM} and~\ref{s:outline_and_motivation} such as such
as \ttbar+\W,\Z or X, stop pair production or the existence of a composite top-antitop top field as postulated
by the topcolour theory.

Existing analyses using $\roots=7\TeV$ data investigating \met and using $\roots=8\TeV$ data investigating all
the global variables listed above can be found in \cite{CMS-PAS-TOP-12-019} and \cite{CMS-PAS-TOP-12-042} respectively.
The combination of the data from both $\roots=7\TeV$ and $\roots=8\TeV$ in this analysis with one single set
of methods for both, allows the comparison of normalised differential cross sections between the
two centre-of-mass-energies. Potential further work related to this analysis may look at a ratio of the
differential cross sections between $\roots=7\TeV$ and $\roots=8\TeV$. Such a measurement would cancel many
systematic measurements, producing a potentially precise measurement.

This analysis forms part of the output of the CMS top quark group~\cite{CMS_top_results}, and complements
similar analyses within CMS investigating top pair cross sections in various other final states such as the
hadronic channel~\cite{Khachatryan:2015fwh}, in which both \W bosons from the top quarks decay hadronically,
leading to a six-jet event signature, and the dileptonic channel~\cite{Chatrchyan:2013faa}, in which both \W
bosons decay leptonically. The technique of top reconstruction, which is not carried out in the analysis
presented in this thesis, allows for the identification of the specific top quark from which the decay
products originate. Top reconstruction is utilised in CMS studies investigating the top pair cross section as
a function of variables such as kinematic properties (like \pt and $\eta$) of the top quarks, \ttbar pair and
charged lepton(s) and \ttbar invariant mass in addition to variables related to the \bjets and jets in the
\ttbar system~\cite{Khachatryan:2015oqa}. Differential cross sections also 

Normalised differential cross section measurements have also been performed by the ATLAS collaboration
at $\roots=7\TeV$~\cite{Aad:2014zka} and $\roots=8\TeV$~\cite{Aad:2015mbv}. While a direct comparison to
ATLAS results is not possible since different variables are used and bin selection is not identical between
analyses. However, the general observations of a harder distribution in predictions than observed in data, and
generator performance such as the good agreement of \MCATNLOHERWIG generated samples with data, have also been
noted in ATLAS analyses.

Run 2 of the LHC, after Long Shutdown 1, began in June 2015 with proton-proton collisions occurring at
$\roots=13\TeV$. Further measurements of \ttbar events are on-going at the LHC on Run 2 data and are an
important component of the future CMS and LHC physics programs. An Early Analysis (EA, analyses aiming to
produce early results and to demonstrate that the detector and simulations are well understood at this early
stage of Run 2) has been carried out on 71\pbinv of LHC Run 2 data from CMS, by the same group that worked on
the analysis presented in this thesis. As collision energies and luminosities at the LHC increase, the
resulting higher statistics and larger cross sections in the coming years could lead to the observation of
rare physics processes and/or the production of potential heavier particles than observed to date.

Run 2 is scheduled to continue until Long Shutdown 2 in 2018 when major accelerator and experiment upgrades
will take place. Until then, regular technical stops, such as during vacation periods, will allow for routine
maintenance to be carried out.
