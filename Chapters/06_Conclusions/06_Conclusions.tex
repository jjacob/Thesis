\chapter{Summary }
\label{c:summary}

This thesis has presented an overview of the theoretical background to the Standard Model, a summary of the
CMS detector at the LHC, and a measurement of differential \ttbar cross sections with respect to global
variables \met, \HT, \st, \mt and \wpt in proton-proton collisions with 5.0~\fbinv of data at \roots=7\TeV
and with 19.7~\fbinv of data at \roots=~8\TeV collected with the CMS experiment at the LHC. 

The main objective of these measurements is to verify the models and generators used to produce simulations of
the signal and background events in CMS. This understanding provides a good basis in new physics analyses
where such events constitute a significant background. In addition, the distributions under investigation
would be sensitive to rare standard model processes; for example, the \met or \mt distributions would be
sensitive to \ttbar+\Z/\W processes, while the \HT, \st and \wpt distributions would provide information on
\ttbar+X production where X is massive and decays to hadrons. Hints of new physics scenarios such as stop pair
production may also be visible in the distributions of global variables.

The results of this analysis confirmed the previously observed characteristic of a \pt distribution that is
softer in data than in the simulation. The simulated distribution corrected for this mismodelling shows good
agreement with data, however. Otherwise, the data shows good general agreement with the theoretical
predictions, showing that these commonly used Monte Carlo simulation generators can be used with confidence to
model \ttbar events.

Run 2 of the LHC after Long Shutdown 1 began in June 2015 with proton-proton collisions occurring at
$\roots=13~\TeV$. Further measurements of \ttbar events are on-going at the LHC on Run 2 data and will no
doubt continue to do so. Currently an Early Analysis (analyses aimed to obtain and demonstrate that the
detector and simulations are well understood at this early stage of Run 2) is being carried out by the same
group that worked on the analysis presented in this thesis on 40~\pbinv of LHC Run 2 data from CMS. As
collision energies and luminosities at the LHC increase, the resulting higher statistics and larger
cross sections in the coming years could lead to the observation of rare physics processes and/or the
production of potential heavier particles than observed to date.

Run 2 is scheduled to continue until Long Shutdown 2 in 2018 when major accelerator and experiment upgrades
will take place. Until then, regular technical stops, such as vacation periods, will allow for routine
maintenance to be carried out. A summary of the service work carried out by the author for the CMS experiment
has also been covered. Work relating to strip tracker operations and maintenance, and investigation of the new
binary CBC readout chip for the strip tracker. Currently the newer CBC2 is undergoing testing, with a CBC3
already in the design stages and final testing to begin in 2018 for a scheduled installation in CMS at the
HL-LHC from 2023 onwards.
