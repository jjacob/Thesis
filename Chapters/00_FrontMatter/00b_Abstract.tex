\begin{abstract}
\thispagestyle{plain} % needed to print page number on abstract page
%This is my last 3.5 years. Actually, make that 4.

The CMS detector at the LHC has collected proton-proton data since 2010, enabling analysts to study the
properties of the top quark, the heaviest known fundamental particle, in great detail. Starting with a
description of the CMS detector and an overview of the Standard Model of particle physics, this thesis goes on
to present a measurement of normalised differential cross sections of \ttbar production in the electron+jets
and muon+jets channels with respect to global event variables.  5.0\fbinv of proton-proton collision data
collected with the CMS experiment at \roots=7\TeV, and with 19.7\fbinv of data collected at \roots=8\TeV is
used. The measurement is performed in bins of the following global event variables: missing transverse energy
(\met), the scalar sum of jet transverse momenta (\HT), the scalar sum of the transverse momenta of all
objects in the event (\st), the transverse momentum (\wpt) and the transverse mass (\mt) of the leptonically
decaying \W boson from the \ttbar decay.

The datasets used in the electron channel were obtained using an electron+jets trigger in 2011 and a single
electron trigger in 2012. Muon datasets were obtained using a single muon trigger during both data taking
periods. Following the application of selection criteria on the datasets and Monte Carlo simulation to obtain
a \ttbar signal sample, and background samples from single top, \W/\ZpJets and QCD multi-jet processes, a fit
of the simluated samples to data is carried out in each bin and in each channel to obtain a \ttbar event
yield. The number of signal events is then unfolded to remove detector and selection effects. Combining the
result from the two channels, the calculation of the normalised differential cross sections is performed.
Comparison with common Monte Carlo generators, including with different modelling parameters, confirms
previous CMS findings of a softer measured transverse momentum distribution than in simulations, but otherwise
there is general consistency between generator predictions and data.

\end{abstract}